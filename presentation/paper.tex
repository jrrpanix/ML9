%%%%%%%%%%%%%%%%%%%%%%%%%%%%%%%%%%%%%%%%%%%%%%%%%%%%%%%%%%%%%%%%%%%%%%%%%%%
%
% ML Presentation Paper 
% Jeremy Lao/John Reynolds
% JJL359, JR4716
%%%%%%%%%%%%%%%%%%%%%%%%%%%%%%%%%%%%%%%%%%%%%%%%%%%%%%%%%%%%%%%%%%%%%%%%%%%

\documentclass{article}

% AMS packages:
\usepackage{amsmath, amsthm, amsfonts}


%-----------------------------------------------------------------
\title{Predicting FOMC Actions using ML and NLP}
\author{
        Jeremy Lao - jjl359 \\
        NYU Computer Science \\
            \and
        John Reynolds - jr4716 \\
        NYU Computer Science \\
}

\begin{document}
\maketitle

\abstract{Add abstract text here }

\section{Introduction}
In this paper we methods in Natural Language Processing and Machine Learning to predict Federal Open Market 
Committee Rate actions (hold or change) using text from Federal Reserve Meeting Minutes, speeches and statements 
from Federal Reserve Officials.  Since this is work in sentimite analysis we created a simulator to 
gain insight into the sensitiviy of detecting sentimite by randomly generating words with from a mix
of distributions containing positive words, negative words and commonly used words.  That simulator
showed accuracy could hit levels of 1.0 given enough training data and high percentage of sentimite words
to overall text.



\subsection{Subsection}\label{sec:nothing}

To add

\subsubsection{Subsubsection}\label{sec:nothing2}

To add

% Bibliography
%-----------------------------------------------------------------
\begin{thebibliography}{99}

\bibitem{Cd94} Hansen,Stephen, McMachon, Michael  \emph{Transparency and Deliberation with the FOMC:a Computational Linquistics Approach}, CEPR Discussion Paper, (2014)

\end{thebibliography}

\end{document}
